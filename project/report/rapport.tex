%%%%%%%%%%%%%%%%%%%%%%%%%%%%%%%%%%%%%%%%%
% Programming/Coding Assignment
% LaTeX Template
%
% This template has been downloaded from:
% http://www.latextemplates.com
%
% Original author:
% Ted Pavlic (http://www.tedpavlic.com)
%
% Note:
% The \lipsum[#] commands throughout this template generate dummy text
% to fill the template out. These commands should all be removed when
% writing assignment content.
%
% This template uses a Perl script as an example snippet of code, most other
% languages are also usable. Configure them in the "CODE INCLUSION
% CONFIGURATION" section.
%
%%%%%%%%%%%%%%%%%%%%%%%%%%%%%%%%%%%%%%%%%

%----------------------------------------------------------------------------------------
%	PACKAGES AND OTHER DOCUMENT CONFIGURATIONS
%----------------------------------------------------------------------------------------

\documentclass[11pt, a4paper]{article}

\usepackage{fancyhdr} % Required for custom headers
\usepackage{lastpage} % Required to determine the last page for the footer
\usepackage{extramarks} % Required for headers and footers
\usepackage[usenames,dvipsnames]{color} % Required for custom colors
\usepackage{graphicx} % Required to insert images
\usepackage{listings} % Required for insertion of code
\usepackage[utf8]{inputenc}
\usepackage[T1]{fontenc}
\usepackage{stmaryrd}
\usepackage{algpseudocode}
\usepackage[frenchb]{babel}
\usepackage{dsfont}
\usepackage{amsmath}
\usepackage{times}
% Margins
\topmargin=-0.45in
\evensidemargin=0in
\oddsidemargin=0in
\textwidth=6.5in
\textheight=9.0in
\headsep=0.25in

\linespread{1.1} % Line spacing

% Set up the header and footer
\pagestyle{fancy}
\lhead{\hmwkAuthorName} % Top left header
\chead{\hmwkTitle} % Top center head
\rhead{\firstxmark} % Top right header
\lfoot{\lastxmark} % Bottom left footer
\cfoot{} % Bottom center footer
\rfoot{Page\ \thepage\ sur\ \protect\pageref{LastPage}} % Bottom right footer
\renewcommand\headrulewidth{0.4pt} % Size of the header rule
\renewcommand\footrulewidth{0.4pt} % Size of the footer rule

\setlength\parindent{0pt} % Removes all indentation from paragraphs

%----------------------------------------------------------------------------------------
%	CODE INCLUSION CONFIGURATION
%----------------------------------------------------------------------------------------

\lstloadlanguages{C} % Load Perl syntax for listings, for a list of other languages supported see: ftp://ftp.tex.ac.uk/tex-archive/macros/latex/contrib/listings/listings.pdf
\lstset{texcl=true, columns=flexible,basicstyle=\small\ttfamily}

\lstdefinestyle{ccode}{language=C, % Use Perl in this example
        frame=single, % Single frame around code
        basicstyle=\small\ttfamily, % Use small true type font
        keywordstyle=[1]\color{Blue}, % Perl functions bold and blue
        keywordstyle=[2]\color{Green}, % Perl function arguments purple
        keywordstyle=[3]\color{Blue}, % Custom functions underlined and blue
        identifierstyle=, % Nothing special about identifiers
        commentstyle=\small\color{Brown}, % Comments small dark green courier font
        stringstyle=\color{OliveGreen}, % Strings are purple
        showstringspaces=false, % Don't put marks in string spaces
        tabsize=5, % 2 spaces per tab
        %
        % Put standard Perl functions not included in the default language here
        morekeywords={f, sequantial_computation, parallel_computation, printf},
        morecomment=[l][\color{Blue}]{...}, % Line continuation (...) like blue comment
        numbers=left, % Line numbers on left
        firstnumber=1, % Line numbers start with line 1
        numberstyle=\tiny\color{Blue}, % Line numbers are blue and small
        stepnumber=5, % Line numbers go in steps of 5
        texcl=true,
        columns=flexible
}

% Creates a new command to include a perl script, the first parameter is the filename of the script (without .pl), the second parameter is the caption
\newcommand{\cscript}[2]{
\begin{itemize}
\item[]\lstinputlisting[caption=#2,label=#1]{#1.pl}
\end{itemize}
}


\newcommand{\problemAnswer}[1]{ % Defines the problem answer command with the content as the only argument
\noindent\framebox[\columnwidth][c]{\begin{minipage}{0.98\columnwidth}#1\end{minipage}} % Makes the box around the problem answer and puts the content inside
}

%----------------------------------------------------------------------------------------
%	NAME AND CLASS SECTION
%----------------------------------------------------------------------------------------

\newcommand{\hmwkTitle}{Projet final : Parallélisation de l'algorithme de Jacobi} % Assignment title
\newcommand{\hmwkClass}{High Performance Computing} % Course/class
\newcommand{\hmwkAuthorName}{Rémi Garde} % Your name

%----------------------------------------------------------------------------------------
%	TITLE PAGE
%----------------------------------------------------------------------------------------

\title{
\LARGE{\textbf{\hmwkClass}}\\
\vspace{0.5in}
\large{\textbf{\hmwkTitle}}
\vspace{3in}
}

\author{\textbf{\hmwkAuthorName}}
\date{Mars 2018} % Insert date here if you want it to appear below your name

%----------------------------------------------------------------------------------------

\begin{document}

\maketitle

%----------------------------------------------------------------------------------------
%	TABLE OF CONTENTS
%----------------------------------------------------------------------------------------

%\setcounter{tocdepth}{1} % Uncomment this line if you don't want subsections listed in the ToC

\newpage
\tableofcontents
\newpage

%----------------------------------------------------------------------------------------
%	PROBLEM 1
%----------------------------------------------------------------------------------------

% To have just one problem per page, simply put a \clearpage after each problem

\section{Pre-Processing}

\subsection{Définition du problème}

Soit $n \in \mathds{N}$, $A = (a_{ij})_{(i,j)\in \llbracket 1,n \rrbracket^2}$ une matrice carrée de taille $n$, $B = (b_{i})_{i\in \llbracket 1,n \rrbracket}$ un vecteur de taille $n$.
Nous cherchons à résoudre le système d'équations linéaires
\begin{equation} \label{eq:problem}
    AX=B
\end{equation}
pour $X = (x_{i})_{i\in \llbracket 1,n \rrbracket}$ vecteur de taille $n$.

\subsection{Résolution}

Pour résoudre ce problème, $A$ est séparée en 2 matrices $D$ et $R$, avec $D$ matrice de la diagonale de $A$ et $R$ les éléments non diagonaux de $A$ :
\[
A =
 \begin{pmatrix}
  a_{1,1} & a_{1,2} & \cdots & a_{1,n} \\
  a_{2,1} & a_{2,2} & \cdots & a_{2,n} \\
  \vdots  & \vdots  & \ddots & \vdots  \\
  a_{n,1} & a_{n,2} & \cdots & a_{n,n}
 \end{pmatrix}
 =
  \begin{pmatrix}
  a_{1,1} & 0 & \cdots & 0 \\
  0 & a_{2,2} & \cdots & 0 \\
  \vdots  & \vdots  & \ddots & \vdots  \\
  0 & 0 & \cdots & a_{n,n}
 \end{pmatrix}
 +
  \begin{pmatrix}
  0 & a_{1,2} & \cdots & a_{1,n} \\
  a_{2,1} & 0 & \cdots & a_{2,n} \\
  \vdots  & \vdots  & \ddots & \vdots  \\
  a_{n,1} & a_{n,2} & \cdots & 0
 \end{pmatrix}
\]
Ceci est intéressant car le calcul de $D^{-1}$ est trivial:
\[
D^{-1}
  \begin{pmatrix}
  \frac{1}{a_{1,1}} & 0 & \cdots & 0 \\
  0 & \frac{1}{a_{2,2}} & \cdots & 0 \\
  \vdots  & \vdots  & \ddots & \vdots  \\
  0 & 0 & \cdots & \frac{1}{a_{n,n}}
 \end{pmatrix}
\]
La formulation \eqref{eq:problem} devient donc équivalente à \((D+R)X=B\)
ce qui donne:
\begin{equation} \label{eq:reeq}
    X = D^{-1} (B-RX)
\end{equation}

Nous pouvons ensuite poser la suite $X^{(k)}, n \in \mathds{N}$ définie comme suit:

\[
\left\{
    \begin{array}{lr}
        X^{(0)} = 0 \\
        X^{(k+1)} = D^{-1} (B-RX^{(k)}),& k \in \mathds{N}^*
    \end{array}
\right.
\]

La formule de récurrence s'écrit pour chacun des éléments de X :

\begin{equation}
    \forall k \in \mathds{N}^*, \forall i \in \llbracket 1,n \rrbracket,
    x^{(k+1)}_i = \frac{1}{a_{ii}} (b_i - \sum_{i \neq j} a_{ij}x^{(k)}_j)
\end{equation}

\subsection{Convergence}

Déterminons une condition tel que \(\lim_{k \to \infty} X^{(k)} = X\)).
En posant \( A^\prime = -D^{-1}R\) et \( B^\prime = D^{-1}B \)

\eqref{eq:reeq} devient alors :

\[
    X = A^\prime X + B^\prime
\]
et ainsi

\[
    X^{(k+1)} - X = A^\prime X^{(k)} + B^\prime - (A^\prime X + B^\prime)
    =  A^\prime (X^{(k)} - X)
\]
ce qui indique
\[
    \lim_{k \to \infty} \| X^{(k+1)} - X \| = 0 \Leftrightarrow \rho(A^\prime) < 1
\]

On remarque que les coefficients de \( A^\prime \) s'écrivent:
\[
    a^\prime_{i,j} = \left\{
        \begin{array}{lr}
            0 & \text{si } i = j \\
            \frac{-a_{ij}}{a_{ii}} & \text{si } i \neq j
        \end{array}
    \right.
\]

Soit \( \lambda \) une valeur propre de \( A^\prime \) et \( Y = (y_i) \)
un vecteur propre associé.
\( A^\prime Y = \lambda Y \) donne donc pour tout \(i\) :
\begin{align*}
    |\lambda y_i| &= | \sum_j a^\prime_{ij} y_j | \\
    |\lambda y_i| &= | \sum_{j \neq i} \frac{-a_{ij}}{a_{ii}} y_j | \\
    |\lambda y_i| &= |\frac{1}{a_{ii}} y_i| \cdot | \sum_{j \neq i} -a_{ij}| \\
    |\lambda y_i| &\leq |\frac{1}{a_{ii}} y_i| \sum_{j \neq i} |a_{ij}|
\end{align*}

Or, pour toute matrice \( M = (m_{ij}) \) à diagonale strictement dominante :
\[
    \forall i, |m_{ii}| > \sum_{j \neq i} |m_{ij}|
\]

Ainsi, si \( A \) est diagonale strictement dominante, alors son rayon spectral est inférieur à 1, ce qui fait converger la méthode de Jacobi.


\section{Processing}
\subsection{Parallélisation}


\section{Post-Processing}

\end{document}
