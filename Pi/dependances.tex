%%%%%%%%%%%%%%%%%%%%%%%%%%%%%%%%%%%%%%%%%
% Programming/Coding Assignment
% LaTeX Template
%
% This template has been downloaded from:
% http://www.latextemplates.com
%
% Original author:
% Ted Pavlic (http://www.tedpavlic.com)
%
% Note:
% The \lipsum[#] commands throughout this template generate dummy text
% to fill the template out. These commands should all be removed when
% writing assignment content.
%
% This template uses a Perl script as an example snippet of code, most other
% languages are also usable. Configure them in the "CODE INCLUSION
% CONFIGURATION" section.
%
%%%%%%%%%%%%%%%%%%%%%%%%%%%%%%%%%%%%%%%%%

%----------------------------------------------------------------------------------------
%	PACKAGES AND OTHER DOCUMENT CONFIGURATIONS
%----------------------------------------------------------------------------------------

\documentclass{article}

\usepackage{fancyhdr} % Required for custom headers
\usepackage{lastpage} % Required to determine the last page for the footer
\usepackage{extramarks} % Required for headers and footers
\usepackage[usenames,dvipsnames]{color} % Required for custom colors
\usepackage{graphicx} % Required to insert images
\usepackage{listings} % Required for insertion of code
\usepackage{courier} % Required for the courier font
\usepackage{lipsum} % Used for inserting dummy 'Lorem ipsum' text into the template
\usepackage{framed}
\usepackage[utf8]{inputenc}
\usepackage{stmaryrd}
\usepackage{algpseudocode}
\usepackage{amsthm}
\usepackage{dsfont}
% Margins
\topmargin=-0.45in
\evensidemargin=0in
\oddsidemargin=0in
\textwidth=6.5in
\textheight=9.0in
\headsep=0.25in

\linespread{1.1} % Line spacing

% Set up the header and footer
\pagestyle{fancy}
\lhead{\hmwkAuthorName} % Top left header
\chead{\hmwkTitle} % Top center head
\rhead{\firstxmark} % Top right header
\lfoot{\lastxmark} % Bottom left footer
\cfoot{} % Bottom center footer
\rfoot{Page\ \thepage\ sur\ \protect\pageref{LastPage}} % Bottom right footer
\renewcommand\headrulewidth{0.4pt} % Size of the header rule
\renewcommand\footrulewidth{0.4pt} % Size of the footer rule

\setlength\parindent{0pt} % Removes all indentation from paragraphs

%----------------------------------------------------------------------------------------
%	CODE INCLUSION CONFIGURATION
%----------------------------------------------------------------------------------------

\definecolor{MyDarkGreen}{rgb}{0.0,0.4,0.0} % This is the color used for comments
\lstloadlanguages{C} % Load Perl syntax for listings, for a list of other languages supported see: ftp://ftp.tex.ac.uk/tex-archive/macros/latex/contrib/listings/listings.pdf
\lstset{language=C, % Use Perl in this example
        frame=single, % Single frame around code
        basicstyle=\small\ttfamily, % Use small true type font
        keywordstyle=[1]\color{Blue}\bf, % Perl functions bold and blue
        keywordstyle=[2]\color{Purple}, % Perl function arguments purple
        keywordstyle=[3]\color{Blue}\underbar, % Custom functions underlined and blue
        identifierstyle=, % Nothing special about identifiers
        commentstyle=\usefont{T1}{pcr}{m}{sl}\color{MyDarkGreen}\small, % Comments small dark green courier font
        stringstyle=\color{Purple}, % Strings are purple
        showstringspaces=false, % Don't put marks in string spaces
        tabsize=5, % 2 spaces per tab
        %
        % Put standard Perl functions not included in the default language here
        morekeywords={f, sequantial_computation, parallel_computation},
        morecomment=[l][\color{Blue}]{...}, % Line continuation (...) like blue comment
        numbers=left, % Line numbers on left
        firstnumber=1, % Line numbers start with line 1
        numberstyle=\tiny\color{Blue}, % Line numbers are blue and small
        stepnumber=5 % Line numbers go in steps of 5
}

% Creates a new command to include a perl script, the first parameter is the filename of the script (without .pl), the second parameter is the caption
\newcommand{\cscript}[2]{
\begin{itemize}
\item[]\lstinputlisting[caption=#2,label=#1]{#1.pl}
\end{itemize}
}


\newcommand{\problemAnswer}[1]{ % Defines the problem answer command with the content as the only argument
\noindent\framebox[\columnwidth][c]{\begin{minipage}{0.98\columnwidth}#1\end{minipage}} % Makes the box around the problem answer and puts the content inside
}

%----------------------------------------------------------------------------------------
%	NAME AND CLASS SECTION
%----------------------------------------------------------------------------------------

\newcommand{\hmwkTitle}{Théorèmes sur les dépendances d'instructions} % Assignment title
\newcommand{\hmwkClass}{High Performance Computing} % Course/class
\newcommand{\hmwkAuthorName}{Rémi Garde} % Your name
%----------------------------------------------------------------------------------------
%	TITLE PAGE
%----------------------------------------------------------------------------------------

\title{
\LARGE{\textbf{\hmwkClass}}\\
\vspace{0.5in}
\large{\textbf{\hmwkTitle}}
\vspace{3in}
}

\author{\textbf{\hmwkAuthorName}}
\date{Février 2018} % Insert date here if you want it to appear below your name

\newtheorem*{theorem}{Théorème}
%----------------------------------------------------------------------------------------

\begin{document}

\maketitle
% \newpage
% \tableofcontents
\newpage

\subsection*{Parallélisation d'une boucle}


\begin{theorem}
Une boucle de programme est parallélisable si et seulement si deux instructions associées à des itérations différentes ne présentent aucune dépendances de données ou de sorties.
\end{theorem}

\begin{proof}[Preuve]
Soit $B$ une boucle de programme . Par commodité, à chaque itération $i \in \llbracket 1,n \rrbracket$ de $B$ nous associons une instruction unique $I_i$, combinaison de toutes les instructions de cette itération.
Montrons que $B$ est parallélisable si et seulement si $I_i$ et $I_j$ n'ont aucunes dépendance de données ou de sorties pour $i \neq j$.

$\Rightarrow$ Supposons $B$ parallélisable, i.e. son output ne dépend pas de l'ordre dans lequel l'itération est effectuée.

Supposons qu'il existe une dépendance de données: $\exists (i,j), In(I_i) \cap Out(I_j) \neq \emptyset$.
Dans ce cas,$Out(I_i)$ va dépendre de $In(I_i)$, donc de $Out(I_j)$.
La sortie de $I_i$ dépend de l'exécution ou non de $I_j$ : c'est impossible.

Supposons qu'il existe une dépendance de sorties. $\exists (i,j), Out(I_i) \cap Out(I_j) \neq \emptyset$.
Dans ce cas, la sortie de $I_i$ sera (au moins en partie) écrasée par l'exécution de $I_j$, et réciproquement. On ne eut intervertir leur ordre sans changer la sortie globale. C'est impossible de même: il ne peut y avoir ni dépendance d'entrées ni de sorties.
\\
$\Leftarrow$ Supposons qu'il n'y ait aucune dépendance dans $B$.
Soit $o$ un output : $o \in \bigcup_{i} Out(I_i)$.
$\exists! k, o \in Out(I_k)$.
Par indépendance de sorties, $k$ est unique.
L'état de $o$ en sortie de $B$ ne dépend donc que $In(I_k)$.
Par indépendance de données, $In(I_k)$ ne dépend d'aucune sortie des autres instructions. Donc $o$ ne dépend pas de l'exécution d'une autre instruction, i.e. $o$ ne dépend pas de l'ordre d'exécution.
$B$ est donc parallélisable.
\end{proof}

\begin{theorem}
Les deux boucles imbriquées d'indices $i$ et $j$ sont permutables si et seulement si il n'existe pas de dépendances de données ou de sorties entre des instances $(i+k_i, j-k_j)$ ou $(i-k_i, j+k_j)$ et l'instance $(i,j)$, $(k_i, k_j) \in \mathds{N}^*$
\end{theorem}

\begin{proof}[Preuve]
La permutation des boucles ne va changer que partiellement l'ordre d'exécution. Pour chaque instance $(i,j)$, on note
\begin{itemize}
\item $A_{1}(i,j)$ les instances exécutées après l'instance $(i,j)$ lorsque l'on utilise l'ordre d'exécution $(i,j)$ : ce sont les instances $(p,q)$ avec $p>i$ ou $p = i, q >j$
\item $A_{2}(i,j)$ les instances exécutées après l'instance $(i,j)$ lorsque l'on utilise l'ordre d'exécution $(j,i)$ ce sont les instances $(p,q)$ avec $q>j$ ou $p > i, q = j$
\item $B_{1}(i,j)$ les instances exécutées avant l'instance $(i,j)$ lorsque l'on utilise l'ordre d'exécution $(i,j)$ : ce sont les instances $(p,q)$ avec $p<i$ ou $p = i, q <j$
\item $B_{2}(i,j)$ les instances exécutées avant l'instance $(i,j)$ lorsque l'on utilise l'ordre d'exécution $(j,i)$ ce sont les instances $(p,q)$ avec $q<j$ ou $p < i, q = j$
\end{itemize}

On peut permuter si et seulement si, pour chaque instance $(i,j)$, il n'y a aucune dépendance de données ou de sorties entre $(i,j)$ et les instances qui changent d'ordre avec $(i,j)$ lors de la permutation.
Ces instances sont $A_{1}(i,j) \cap B_{2}(i,j)$ et $A_{2}(i,j) \cap B_{1}(i,j)$.

Considérons l'instance $(p,q)$, $I_{p,q} \in A_{1}(i,j) \cap B_{2}(i,j)$.

$$
\left\{
    \begin{array}{ll}
        p > i & \mbox{ou } p = i, q > j \\
        q < j & \mbox{ou } p < i, q = j
    \end{array}
\right.
\mathrm{soit} ~~
\left\{
    \begin{array}{l}
        p > i  \\
        q < j
    \end{array}
\right.
\mathrm{ou~encore} ~~
\exists (k_i, k_j) \in \mathds{N}^*, (p,q) = (i+k_i, j-k_j)
$$

De la même façon, $A_{2}(i,j) \cap B_{1}(i,j)$ donne $\exists (k_i, k_j) \in \mathds{N}^*, (p,q) = (i-k_i, j+k_j)$.

Donc les boucles sont permutables si et seulement si, $\forall (i,j), \forall (k_i, k_j) \in \mathds{N}^*$, il n'y a aucune  indépendance de données entre les instances $(i,j)$ et $(i-k_i, j+k_j)$ ou $(i+k_i, j-k_j)$.

\end{proof}

\end{document}
